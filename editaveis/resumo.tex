\begin{resumo}
 Atualmente, a música está num patamar único no que diz respeito às várias abordagens de se contemplar e se executar e, com isso, a tecnologia vem cada vez mais sendo usada para otimizar os processos musicais. Um dos exemplos de tecnologia são sistemas automáticos de transcrição de música que auxiliam o músico, substituindo por vezes de maneira significativa partituras, tablaturas e cifras. Esse presente trabalho tem como objetivo desenvolver um protótipo de uma solução computacional para reconhecimento de harmonias musicais. Para tal fim, priorizou-se a implementação da análise espectral da amostra de áudio, classificação em notas musicais, classificação em acordes com suportes a inversões, transição rítmica, reconhecimento dos padrões harmônicos ao longo do tempo e extração de tonalidade musical. O desenvolvimento da solução se deu através de um método de desenvolvimento empírico, iterativo e incremental, utilizando a linguagem de programação Matlab para implementação. De fundamentos teóricos foram utilizados conceitos físicos do som, teoria musical, processamento de sinais e redes neurais artificiais. O desenvolvimento da solução permitiu o reconhecimento de acordes em tríades maiores, menores, aumentados, diminutos e invertidos em amostras isoladas de acordes gravados, transcrição automática de acordes ao longo do tempo e extração de tonalidade musical.

 \vspace{\onelineskip}

 \noindent
 \textbf{Palavras-chaves}: reconhecimento. acordes. música. processamento. sinais. redes. neurais. harmonia.
\end{resumo}
