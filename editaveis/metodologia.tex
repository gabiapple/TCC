\chapter{Metodologia}
\label{chap:metodologia}

\section{Desenvolvimento Empírico, Iterativo e Incremental}
\label{sec:empirico}

Esse trabalho está dentro de um contexto específico na engenharia de software. Esse contexto se caracteriza num projeto de pesquisa e desenvolvimento no qual até a consolidação da solução pouco se sabia sobre sua natureza (por exemplo arquitetura e viabilidade), o andmento do desenvolvimento ocorreu de forma não-linear e complexa e houve um frequente diálogo entre disciplinas de diferentes domínios - transdiciplinaridade.

No que diz respeito ao desconhecimento da natureza da solução no início do projeto, houve várias questões teóricas de pesquisa que foram passíveis de experimentações e disscusões ao longo do processo de desenvolvimento. Tal fato ocasionou na caracterização da solução ao longo do projeto e sua definição final somente no fim.





abordagem não-linear e complexa.

A não-linearidade   produção intelectual variou muito


a transdisciplinaridade. Esse termo é empregado para o último patamar que conhecemos hoje sobre as relações entre disciplinas e domínios de conhecimento. Conceitua-se como tal um processo que transcende as disciplinas, que está entre, além e através das disciplinas \cite{criatividade}. Nesse trabalho foram investigados vários domínios de conhecimento como a engenharia, matemática, neurociências, psicologia, música e computação.




\section{Linguagem de Programação}
\label{sec:linguagemprogramacao}

Existem algumas ferramentas para muitas linguagens de programação que focam processamento de sinais. Uma delas é uma biblioteca em C++ desenvolvida por David Weenink baseada no processomento Mel Frequency Cepstral Coefficients (MFCC) \footnote{http://kaldi.sourceforge.net/feat.html\#feat\_mfcc}. Ela dá suporte a extração de dados de arquivos, cálculo da transformada de fourier (FFT), seu respectivo espectro de energia e cálculo da transformada de cosseno. Uma outra vantagem do uso dela é a linguagem de programação C++ que é bastante rápida em relação ao Java, Ruby e Python. Porém ela não dá suporte a estruturas de álgebra linear e cálculos estatísticos como operação de correlação. Processamento de sinais e redes neurais demanda também muito esforço no uso de matrizes e suas respectivas operações, fato esse torna o uso da linguagem C++ e da biblioteca citada desvantajosos.

Em vista das necessidades expostas, a linguagem de programação/ferramental escolhida para o trabalho é o Matlab \footnote{http://www.mathworks.com/}. Matlab é um software científico para computação numérica. Essa escolha foi feita primeiramente por tratar de ser um software voltado para aplicações científicas e para os axiomas da álgebra linear. Além disso, essa plataforma possui um conjunto de ferramentas para visualização de gráficos, pacotes de fórmulas matemáticas pré-programadas e estruturas de dados voltadas para a análise numérica e matricial. Ela possui uma característica de ser interpretada e dinamicamente tipada.
