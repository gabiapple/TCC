\begin{agradecimentos}

Agradeço primeiramente a Deus, inteligência criadora suprema, por permitir-me nesse mundo, vivendo, aprendendo e contemplando a beleza da natureza primordial de todas as coisas.

A minha amada e querida mãe Francisca, pela paciência, compreensão, tolerância, conselhos, carinho, dedicação, afeto, amizade, silêncio, sorrisos e um intenso amor. Meu primeiro aprendizado na vida mais puro e original de amor foi através dela. Isso me possibilitou a amar verdadeiramente o que faço e ter uma visão de vida mais profunda. Meus sinceros e eternos agradecimentos.

A meu pai Luciano, mesmo não estando presente mais, me inspirou a escolha da minha formação e me ensinou a olhar o mundo com meus próprios olhos.

A meu irmão João, companheiro e amigo de sempre. Seus conselhos e seu exemplo têm me ensinado muito a ser uma pessoa melhor.

A meu padrinho Inácio, meu segundo pai, por seus profundos conselhos sobre a vida e um exemplo para mim de homem honrado e correto.

A meu tio Antônio, por ser o padrinho da minha mãe e assumir o papel de avô na minha vida.

A minha querida tia Naíde, por ter assumido papel de avó na minha vida e ter tido um olhar único sobre minha vida. 

A minha madrinha Nevinha, minhas tias Titia, Tia Marli, Tia Gracinha e Tia Bá. O amor delas é indescritível com palavras. A toda minha família pelo apoio, confiança e compressão.

A meus amigos-irmãos Thiago e Leandro, pelo companheirismo indescritível de muitos anos e apoio de sempre.

A minha amiga Marina Shinzato, pelas longas conversas e por ter sido meu ombro forte ao longo desse trabalho.

A minha amiga Anaely, pelo apoio compreensão e inspiração.

A minha amiga Ana Luisa, pelas longas conversas e incrível amizade. 

A meus professores de música Boggie e Gedeão por todo conhecimento e inspiração musical. 

A meu orientador professor Henrique Moura, pelo exemplo, inspiração, amizade, conselhos, apoio, confiança e investimento de longas conversas. Esse trabalho necessariamente foi fruto de uma orientação em excelência. 

A meu co-orientador professor Paulo Meirelles, pelo exemplo e ensinamentos valiosos e práticos sobre o mundo do software e a vida.

Aos professores Hilmer, Milene, Maria de Fátima, Cristiano e Fernando pelos valiosos ensinamentos e exemplos de profissionais-cientistas.

A equipe do LAPPIS pelo suporte e aprendizado na produção de softwares de qualidade.

A professora Suzete e a equipe do MídiaLab por todo aprendizado.

Aos meus amigos da faculdade e companheiros de disciplinas Carlos, Álvaro, Fagner, Eduardo, Wilton, João, Daniel, Matheus, Kleber, Hebert, André Guedes, David, Yeltsin, Wilker, Thaiane, Tomaz, Maxwell, Luiz Oliveira e  André Mateus, pela compreensão, apoio e motivação.

Aos meus restantes amigos Luiz Matos, Fábio Costa, Daniel Bucher, Renan, Chico, Leônidas, Lucas, Nayron, Thiago Ribeiro, Marcos Ramos, Cleiton, Marcos Ronaldo, José Alisson, José Alberto, Vilmey, Yan, Igor Josafá, Guilherme Fay, Sérgio, Lucas Kanashiro, Charles Oliveira, Rodrigo, Álex, Jefferson, Alexandre, Matheus Souza, Ana Luiza e outros que esqueci de citar, pelo apoio e zueira de sempre.

E as pessoas que passaram na minha vida e influenciaram de alguma forma nesse trabalho. Meus agradecimentos.





\end{agradecimentos}
