\begin{resumo}[Abstract]
 \begin{otherlanguage*}{english}
   Currently, the music have been in top level with regard to various approaches to behold and run. The technology is increasingly becoming too an interaction approach with the musical processes. One of the technology examples are automatic music transcription systems that help the musician, improving significantly scores, tabs and chords. This present study aims to develop a prototype of computational solution for recognition of musical harmonies. For this purpose, implementations of spectral analysis of the audio sample, classification of musical notes, chord classification with support inversion, recognition of rhythmic and harmonic transition patterns over time and extraction of musical tonalities were made. The development of the solution was through a method of empirical, iterative and incremental cicles, using Matlab programming language for implementation. Of theoretical foundations were used physical concepts of sound, music theory, signal processing and artificial neural networks. The development solution has allowed the recognition of the chord triads in larger, smaller, increased, and miniature inverted in isolated samples of recorded chords, chord automatic transcription over time and extraction of musical tone.

   \vspace{\onelineskip}

   \noindent
   \textbf{Key-words}: recognition. chords. music. processing. signals. networks. neural.
 \end{otherlanguage*}
\end{resumo}
