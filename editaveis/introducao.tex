\chapter[Introdução]{Introdução}
\label{chap:introducao}

Atualmente a música está num patamar único no que diz respeito a várias abordagens de se contemplar e se executar. A tecnologia vem cada vez mais se tornando uma abordagem de interação com os processos musicais \cite{makingmusictechnology}. Desde sintetizadores eletrônicos até afinadores programados em software, a música vem acompanhando o desenvolvimento técnico-científico.

Um bom exemplo de impacto direto da tecnologia sobre a música é o software Auto-Tune \cite{autotune}. O software é um editor de audio em tempo real criado pela empresa Antares Audio Technologies \cite{autotune2} para afinar instrumentos e vozes. Muitos cantores e artistas, especificamente pops, usam desse software para poder executar as músicas com mais afinação em shows e gravações. Exemplos de artistas que usam são Rihanna, Justin Bieber, Demi Lovato, Bruno Mars, Kelly Clarkson e Lady Gaga. De fato milhares de pessoas são impactadas pelo resultado do trabalho desse software.

Outro exemplo de software impactante na música é o afinador Tuner-gStrings \cite{afinador}. Ele é um aplicativo da plataforma para dispositivos móveis Android que permite a afinação de quaisquer instrumentos musicais. Na loja virtual Google Play ele está com 4,6 de 5 estrelas em 155.957 avaliações e o número de instalações entre 10.000.000 e 50.000.000 no mundo inteiro.

Ambos softwares apresentados são ferramentas de suporte para o músico poder executar corretamente as músicas e facilitar muito trabalho que seria de natureza manual. O presente trabalho tem como foco apresentar uma solução computacional de uma possível ferramenta de suporte ao músico.


\section{Problemática}
\label{sec:problematica}

Os músicos em geral sempre necessitaram do conhecimento de informações sobre as músicas com o intuito de serem melhor executadas. Informações do tipo de compasso, tom, harmonia, escalas utilizadas, andamento, expressões e variações de tempo. Especificamente obter a noção de harmonia e tom das músicas é de grande valor no que diz respeito a instrumentos melódicos e jazzistas \cite{jazzistas}. 

Normalmente as informações de harmonia e tom são inferidas na partitura e tablaturas por regras simples como a primeira nota que começa e termina a musica ou como o primeiro acorde que começar e terminar. Também são utilizados noções de escalas e acidentes para inferir que tais notas realmente pertencem a um determinado tom.

Essas técnicas são efetivas se no caso houver partituras, tablaturas ou cifras.Também há a possibilidade, mas somente para quem tem um ouvido bastante treinado, de ouvir melodias e harmonias e poder extrair informações de acordes e tom. Poucas pessoas possuem essa habilidade de discernir notas, tons e harmonia apenas ouvindo o som.

Em vista desse contexto, sistemas automáticos de transcrição de música \cite{automaticmusic} são perfeitamente adequados a atender as necessidades de extração de informações relevantes numa dada faixa de áudio.

No que se trata diretamente sobre acordes, existem poucos estudos publicados sobre um sistema de detecção de acordes musicais. Um dos poucos publicados é baseado num método utilizado em tecnologias 3G CDMA para dispositivos móveis \cite{picchords}. É bastante interessante a correlação que o estudo faz de notas tocadas com clientes CDMA's. Entretanto a técnica CDMA de cancelamento de interferência paralelo (PIC) possui limitações quando se aplica a notas musicais devido a natureza não ortogonal das mesmas (em clientes CDMA são ortogonais entre si, já para notas musicais há o problema dos seus respectivos harmonônicos, que não são ortogonais entre si).

No ponto de vista também da aprendizagem musical há uma motivação para a criação de um detector de acordes, dado que muitos iniciantes não sabem os acordes corretos para cada posição do instrumento, como também, os ouvidos são pouco treinados. Um sistema capaz de auxiliar na detecção das harmonias seria de grande ajuda para o aprendiz abstrair os padrões musicais.

Em visto do que foi exposto, um detector de acordes musicais facilitaria a atuação do músico por informar acordes e notas. Isso é muito bom pois substituiria parcialmente o uso da partitura, tablaturas e cifras além de funcionar como um ótimo guia para solistas e improvisadores (principalmente jazzistas).

\section{Objetivos}
\label{sec:objetivos}

O presente trabalho tem como objetivo principal propor uma solução computacional para detecção de acordes musicais.

Como objetivos específicos têm-se:

 	\begin{itemize}
        \item implementar a transformada de fourier para descrever os sinais de audio em termos de frequências;
        \item implementar uma rede neural de inferência para sugestão de notas;
        \item implementar uma rede neural de inferência para sugestão de acordes.
    \end{itemize}

