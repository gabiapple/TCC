\chapter{Conclusões}
\label{chap:conclusoes}
Desde os primórdios da humanidade, a música vem se desenvolvendo de forma intensa, acoplando vários elementos da cultura corrente aos seus processos de audição, estruturação, composição e execução. Um tipo específico desses elementos são as tecnologias computacionais que, ao interferirem nos processos musicais, otimizam e catalizam as várias formas de se interagir com a música. O produto desse trabalho, então, é um protótipo de uma tecnologia computacional para reconhecimento e extração de informações harmônicas, objetivando a automação da percepção musical.

No ponto de vista da automação do reconhecimento e extração de harmonias musicais, o músico muitas vezes não tem acesso a partituras ou cifras para executar músicas e esse fato se intensifica bastante quando é um estudante iniciante que, muitas vezes, necessita de um auxiliador técnico para orientação. Nesse trabalho foram elaborados e implementados os processos computacionais de reconhecimento de acordes musicais e extração de tonalidades musicais, informações harmônicas essas que são de grande valia para amenizar os problemas citados.

Para a consolidação do protótipo do sistema-solução, objetivou-se desenvolver soluções computacionais para reconhecimento de acordes e suas inversões, reconhecimento de acordes ao longo do tempo e extração do tom da música. O primeiro objetivo foi totalmente alcançado visto que o sistema reconheceu, em amostras separadas de acordes gravados, 100\% das possibilidades de tríades (maiores, menores, aumentadas, diminutas e invertidas). O segundo objetivo foi parcialmente alcançado visto que o sistema reconheceu a sequência de acordes num áudio completo tanto para piano com 75\% de acertos, tanto para violão com 50\% de acertos. O terceiro objetivo foi parcialmente alcançado visto que a solução reconheceu 67\% corretamente o tom de todas as amostras de músicas corretamente, porém vale ressaltar que músicas completas, polifônicas e multi-instrumentais formaram parte das amostras testadas e esse fato significa um grande ganho do protótipo sistema-solução visto que até então os objetivos anteriores se limitaram somente a músicas mono-instrumentais. Esse fato corrobora na afirmação de que o sistema-solução pode obter resultados coerentes em situações mais complexas na extração de tonalidade.

Também é passível de consideração as implementações não satisfatórias para o contexto do problema e melhoria do protótipo sistema-solução. A primeira implementação é a detecção de transições rítmicas, que não foi satisfatória para localizar a ocorrência de acordes ao longo do tempo, devido aos picos de detecção estarem localizados em lugares incoerentes em relação aos locais de execução dos acordes. Pode-se concluir que a detecção de transições rítmicas somente é satisfatória para extrair a quantidade de acordes numa determinada janela de tempo. A segunda implementação é a transformada wavelets que não foi adequada para filtrar frequências específicas de notas musicais devido às distorções nos sinais de saída ocasionadas pelas iterações no banco de filtros.

Esse trabalho atingiu resultados congruentes no que é apresentado no primeiro trabalho da área de computação musical sobre o assunto tratado \cite{fujishima1999realtime}. Todavia, além de atingir resultados semelhantes no que diz respeito a reconhecimento de acordes em amostras gravadas separadamente, o sistema-solução desenvolvido provê parcialmente o reconhecimento de acordes ao longo do tempo num áudio completo e extração de tonalidade em músicas completas, inclusive, multi-instrumentais.

\section{Ameaças e Limitações}

Em vista das ameaças e limitações do trabalho, a análise exposta para o reconhecimento de acordes em amostras gravadas separadamente foi feita somente para o piano, essa análise não foi feita para outros instrumentos musicais. Espera-se que, para outros instrumentos musicais, possa ter alguns resultados divergentes, requerindo novas implementações no sistema-solução a fim de atingir a adaptação a outros instrumentos musicais.

Há ameaças e limitações também no reconhecimento de acordes ao longo do tempo. Na análise feita, o mesmo não atingiu 100\% de acertos pois a implementação de segmentação de áudio não detecta o acorde ao longo do tempo, ela somente sugere uma estimativa de um acorde numa determinada janela de tempo sem considerar a presença ou não de um acorde, ou seja, ruídos de fundo são caracterizados como acordes.

No que diz respeito a extração de tonalidade musical, foi mostrado resultados insatisfatórios tanto para músicas completas multi-instrumentais como para músicas mono-instrumentais (caso do violão). Compreende-se de que ruídos de fundo ocasionam interferências no espectro de frequência, resultando em informações incoerentes para a interpretação da rede neural.

Também há de se questionar as metodologias de avaliação e análise do sistema-solução como um todo. O método quantitativo foi baseado em acertos e erros nos resultados finais do protótipo numa base de músicas relativamente pequena.

\section{Desdobramentos}

Ao decorrer desse trabalho, especificamente ao tratar de uma solução matemática para a substituição da transformada de fourier janelada através da hipótese das transformadas wavelets (seção \ref{sec:wavelets}), houve um desdobramento no trabalho. Como o uso das transformadas wavelets, nesse contexto, não foi satisfatório, foi investigado uma nova adaptação dessa transformada para o problema de identificação de notas musicais.

No que diz respeito aos fundamentos da álgebra linear, a transformada de fourier e as demais, inclusive wavelets, possui uma função primordial que é projetar em bases a informação a ser processada e tratada. A transformada de fourier e wavelets projetam os sinais em bases ortonormais fazendo com que o produto final dessa operação seja coeficientes energéticos distribuídos ao longo de cada componente da base. Na transformada wavelets a operação que faz com que o sinal seja projetado em bases wavelets é a convolução. Dado que projetar o sinal em bases de famílias wavelets não proporcionou resultados satisfatórios, pensou-se na projeção do sinal em bases de notas musicais, ou seja, projetar o sinal, através da operação de convolução, em senos frequências bem definidas somados com seus respectivos harmônicos. Ao final, para analisar se determinadas notas estão presentes no sinal, é plausível de se calcular a energia do sinal para cada nota. Ao final do processo um conjunto de energias de notas serão calculadas, explicitando assim, quais notas foram realmente tocadas.

Há resultados empíricos satisfatórios dessa estratégia de projeção de notas musicais mas tal método carece de uma metodologia rigorosa através de prova matemática e comparações com métodos do estado da arte da computação musical. Esse desdobramento foi submetido ao congresso ISMIR 2015\footnote{http://ismir2015.uma.es/} e, por causa desses pontos citados, não foi aceito e ainda está sendo trabalhado.

\section{Trabalhos Futuros}

Há de se considerar, em trabalhos futuros, análises e testes do protótipo da solução desenvolvida com outros instrumentos musicais. Tais sugestões podem revelar novas hipóteses a serem discutidas e implementadas a fim de melhorar o sistema-solução e simplificar alguns processos.

Também deve-se revisar, discutir e reimplementar o procedimento de segmentação de áudio, pois o mesmo possui falhas não detecção de acordes no momento em que ocorre. Um método mais adequado na segmentação de áudio poderá oferecer melhoridas substanciais no reconhecimento de acordes ao longo do tempo. Para auxiliar essa implementação é plausível de se considerar a inserção de conhecimento na rede neural em casos que não é acorde, fazendo com que o classificador tenha a capacidade de apontar ruídos em geral.

Outra característica que poderia ser implementada e atribuída ao protótipo sistema-solução é um módulo de pré-processamento de áudio, objetivando a amenização de ruídos mesmo contendo cortes em alguns harmônicos do instrumento tocado. Quando mais o áudio se aproximar a tons puros, melhor o sistema-solução irá classificar o áudio.

Em relação às metodologias de avaliação e análise do sistema-solução como um todo, sugere-se o uso das ferramentas disponíveis em MIREX\footnote{http://www.music-ir.org/mirex/wiki/MIREX\_HOME}. Essas ferramentas oferecem um suporte mais sistemático e criterioso na avaliação de sistemas de recuperação musical no geral. As soluções de computação musical de estado da arte são avalidas por esse modo e, fazendo o uso abrangente, pode-se comparar com mais fidedignidade esse e mais outros trabalhos no campo da computação musical.

Como trabalho futuro também há a necessidade de se implementar o projeto proposto, em Python, do protótipo de solução computacional apresentado nesse trabalho. A implementação desse trabalho num projeto de solução de engenharia de software oferecerá ambiente propício para boa manutenibilidade e evolução do sistema. Vários contribuidores poderão trabalhar no sistema visto que há poucas soluções de código aberto sobre as funcionalidades de extração de acordes e tonalidades musicais.